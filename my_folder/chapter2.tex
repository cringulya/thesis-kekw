\chapter{Обзор существующих решений} \label{ch2}

% не рекомендуется использовать отдельную section <<введение>> после лета 2020 года \section{Введение} \label{ch2:intro}

\section{Подходы к физически корректной симуляции} \label{ch2:title-abbr} %название по-русски


Среди самых распространенных подходов к симуляции жидкости можно выделить три
подхода:
\begin{itemize}
	\item Подход Эйлера\\
	      Область симуляции делится на сетку, в каждом элементе считается векторное
	      поле скоростей
	\item Подход Лагранжа\\
	      Жидкость представляется множеством частиц со своей массой, плотностью,
	      скоростью
	\item Heightfield\\
	      Высота поверхности воды представляется некоторой функцией, что уменьшает
	      размерность задачи
\end{itemize}

В предложенном алгоритме предлагается использовать подход Лагранжа, так как
область симуляции не ограничена сеткой, в отличие от подхода Эйлера. А также
симуляция позволяет моделировать брызги, которые невозможны, если уменьшить
размерность задачи как в heightfield подходе.


\section{Теоретические основы SPH} \label{ch2:sec-abbr} %название по-русски

Для симуляции частиц был выбран метод Smoothed Particle Hydrodynamics (SPH).

SPH - метод интерполяции для системы частиц, в котором векторное поле
определеное только в конечном количестве точек, может быть вычислено в любой
точке пространства путем апроксимации.

Скалярная велечинап $A$ может быть интерполирована в точке r как взвешенная
сумма вклада всех остальных частиц.

\begin{equation}
	A_s(r) = \sum_i m_j \frac{A_j}{\rho_j} W(r - r_j, h)
\end{equation}

Здесь $j$ итерирует по всем частицам, $m_j$ - масса частицы $j$, $r_j$ -
положение частицы $j$, $\rho_j$ - плотность, $A_j$ - значение поля в точке $r_j$

Функция $W(r, h)$ называется сглаживающей ядерной функцией с радиусом $h$.


\section{Название параграфа} \label{ch2:sec-very-short-title} %название по-русски



\input{my_folder/tex/eq-equation-multilined} % пример оформления одиночной формулы в несколько строк

\input{my_folder/tex/fig-spbpu-sc-four-in-one} % пример подключения 4х иллюстраций в одном рисунке

%\input{my_folder/tex/fig-spbpu-whitehall-three-in-one} % пример подключения 3х иллюстрации в одном рисунке
%
%\input{my_folder/tex/fig-spbpu-main-bld-two-in-one} % пример подключения 2х иллюстраций в одном рисунке

\input{my_folder/tex/tab-more-than-one-page} % пример подключения таблицы на несколько страциц


\begin{table} [htbp]% Пример оформления таблицы
	\centering\small
	\caption{Пример представления данных для сквозного примера по ВКР \cite{Peskov2004}}%
	\label{tab:ToyCompare}
	\begin{tabular}{|l|l|l|l|l|l|}
		\hline
		$G$   & $m_1$ & $m_2$ & $m_3$ & $m_4$ & $K$ \\
		\hline
		$g_1$ & 0     & 1     & 1     & 0     & 1   \\ \hline
		$g_2$ & 1     & 2     & 0     & 1     & 1   \\ \hline
		$g_3$ & 0     & 1     & 0     & 1     & 1   \\ \hline
		$g_4$ & 1     & 2     & 1     & 0     & 2   \\ \hline
		$g_5$ & 1     & 1     & 0     & 1     & 2   \\ \hline
		$g_6$ & 1     & 1     & 1     & 2     & 2   \\ \hline
	\end{tabular}
	%	\caption*{\raggedright\hspace*{2.5em} Составлено (или/и рассчитано) по \cite{Peskov2004}} %Если проведена авторская обработка или расчеты по какому-либо источнику	
	\normalsize% возвращаем шрифт к нормальному
\end{table}



%% please, before using, read the author guide carefully

\input{my_folder/tex/tab-toy-context-minipage} % пример подключения minipage

\input{my_folder/tex/fig-spbpu-new-bld-autumn-minipage} % пример подключения minipage




\input{my_folder/tex/rules-theorem-like-expressions}

По аналогии с нумерацией формул, рисунков и таблиц нумеруются и иные текстово-графические объекты, то есть включаем в нумерацию номер главы, например: теорема 3.1. для первой теоремы третьей главы монографии. Команды \LaTeX{} выставляют нумерацию и форматирование автоматически. Полный перечень команд для подготовки текстово-графических и иных объектов находится в подробных методических рекомендациях \cite{spbpu-bci-template-author-guide}.


\input{my_folder/tex/rules-list-of-environments} % список некоторых окружений


\input{my_folder/tex/theorem-example} %пример оформления теоремы


\input{my_folder/tex/definition-example} %пример оформления определения


Вместо теоремо-подобных окружений для вставки небольших текстово-графических объектов иногда используются команды. Типичным примером такого подхода является команда \verb|\footnote{text}|\footnote{Внимание! Команда вставляется непосредственно после слова, куда вставляется сноска (без пробела). Лишние пробелы также не указываются внутри команды перед и после фигурных скобок.}, где в аргументе \verb|text| указывают текст \textit{подстрочной ссылки (сноски)}.В них \textit{нельзя добавлять веб-ссылки или цитировать литературу}. Для этих целей используется список литературы. Нумерация сносок сквозная по ВКР без точки на конце выставляется в шаблоне автоматически, однако в каждом приложении к ВКР нумерация, зависящая от номера приложения, выставляется префикс <<П>>, например <<П1.1>> --- первая сноска первого приложения.




%\FloatBarrier % заставить рисунки и другие подвижные (float) элементы остановиться


\section{Выводы} \label{ch2:conclusion}

Текст заключения ко второй главе. Пример ссылок \cite{Article,Book,Booklet,Conference,Inbook,Incollection,Manual,Mastersthesis,Misc,Phdthesis,Proceedings,Techreport,Unpublished,badiou:briefings}, а также ссылок с указанием страниц, на котором отображены те или иные текстово-графические объекты  \cite[с.~96]{Naidenova2017} или в виде мультицитаты на несколько источников \cites[с.~96]{Naidenova2017}[с.~46]{Ganter1999}. Часть библиографических записей носит иллюстративный характер и не имеет отношения к реальной литературе.

Короткое имя каждого библиографического источника содержится в специальном файле \verb|my_biblio.bib|, расположенном в папке \verb|my_folder|. Там же находятся исходные данные, которые с помощью программы \texttt{Biber} и стилевого файла \texttt{Biblatex-GOST} \cite{ctan-biblatex-gost} приведены в списке использованных источников согласно ГОСТ 7.0.5-2008.
Многообразные реальные примеры исходных библиографических данных можно посмотреть по ссылке \cite{ctan-biblatex-gost-examples}.

Как правило, ВКР должна состоять из четырех глав. Оставшиеся главы можно создать по образцу первых двух и подключить с помощью команды \verb|\input| к исходному коду ВКР. Далее в приложении \ref{appendix-MikTeX-TexStudio} приведены краткие инструкции запуска исходного кода ВКР \cite{latex-miktex,latex-texstudio}.

В приложении \ref{appendix-extra-examples} приведено подключение некоторых текстово-графических объектов. Они оформляются по приведенным ранее правилам. В качестве номера структурного элемента вместо номера главы используется <<П>> с номером главы. Текстово-графические объекты из приложений не учитываются в реферате.



%% Вспомогательные команды - Additional commands
%
%\newpage % принудительное начало с новой страницы, использовать только в конце раздела
%\clearpage % осуществляется пакетом <<placeins>> в пределах секций
%\newpage\leavevmode\thispagestyle{empty}\newpage % 100 % начало новой страницы
